% ****** Start of file apssamp.tex ******
%
%   This file is part of the APS files in the REVTeX 4.1 distribution.
%   Version 4.1r of REVTeX, August 2010
%
%   Copyright (c) 2009, 2010 The American Physical Society.
%
%   See the REVTeX 4 README file for restrictions and more information.
%
% TeX'ing this file requires that you have AMS-LaTeX 2.0 installed
% as well as the rest of the prerequisites for REVTeX 4.1
%
% See the REVTeX 4 README file
% It also requires running BibTeX. The commands are as follows:
%
%  1)  latex apssamp.tex
%  2)  bibtex apssamp
%  3)  latex apssamp.tex
%  4)  latex apssamp.tex
%
\documentclass[%
reprint,
%superscriptaddress,
%groupedaddress,
%unsortedaddress,
%runinaddress,
%frontmatterverbose, 
%preprint,
%showpacs,preprintnumbers,
%nofootinbib,
%nobibnotes,
%bibnotes,
amsmath,amssymb,
aps,
%pra,
%prb,
%rmp,
%prstab,
%prstper,
%floatfix,
]{revtex4-1}

\usepackage{graphicx}% Include figure files
\usepackage{dcolumn}% Align table columns on decimal point
\usepackage{bm}% bold math
\usepackage{amsmath}
\usepackage[outdir=./images/]{epstopdf}
\graphicspath{ {./images/} }
\usepackage{subfigure}



\begin{document}
	
\preprint{APS/123-QED}
	
\title{Parliamentary network based on political speech}% Force line breaks with \\
%\thanks{A footnote to the article title}%

%By alphabetical order: last name	
\author{Elisenda Ortiz Castillo}
%\altaffiliation[Also at ]{Physics Department, XYZ University.}%Lines break automatically or can be forced with \\
\author{Ben Curran}%
%\email{Second.Author@institution.edu}
\author{Hyle Higham}
\author{Demival Vasques Filho}
%\affiliation{%
	%Department of Physics, University of Auckland, Private Bag 92019, Auckland, New Zealand\\
	%This line break forced with \textbackslash\textbackslash
%}%
	
%\collaboration{MUSO Collaboration}%\noaffiliation
	
%\author{Charlie Author}
%\homepage{http://www.Second.institution.edu/~Charlie.Author}
%\affiliation{
%Second institution and/or address\\
%This line break forced% with \\
%}%
%\affiliation{
%Third institution, the second for Charlie Author
%}%
%\author{Delta Author}
%\affiliation{%
% Authors' institution and/or address\\
% This line break forced with \textbackslash\textbackslash
%}%
	
%\collaboration{CLEO Collaboration}%\noaffiliation
	
\date{\today}% It is always \today, today,
%  but any date may be explicitly specified
	
\begin{abstract}
%Context/background information
%specific problem/gap
% Method
% Results

  
 
%\begin{description}
%\item[Usage]
%Secondary publications and information retrieval purposes.
%\item[PACS numbers]
%May be entered using the \verb+\pacs{#1}+ command.
%\item[Structure]
%You may use the \texttt{description} environment to structure your abstract;
%use the optional argument of the \verb+\item+ command to give the category of each item. 
%\end{description}
\end{abstract}

%\pacs{Valid PACS appear here}% PACS, the Physics and Astronomy
                             % Classification Scheme.
%\keywords{Suggested keywords}%Use showkeys class option if keyword
                              %display desired
\maketitle

%\tableofcontents
\section{\label{sec:brownsq}Brown's questions}

\begin{enumerate}
	
	\item Who are the intended readers? (3-5 names)
	\begin{enumerate}
	
		\item
	
	\end{enumerate}
	
	\item What did you do? (50 words) \\
	
	
	\item Why did you do it? (50 words) \\

	
	\item What happenened (when you did that – 3)? (50 words) \\ 
	

	\item What do the results mean in theory? (50 words) \\
	

	\item What do the results mean in practice? (50 words) \\
	
	
	\item What is the key benefit for your readers? (25 words) \\
	
	
	\item What remains unresolved? \\
	\begin{enumerate}
		\item
	
	\end{enumerate}		 		 
	
\end{enumerate}
 

Extra questions:

\begin{enumerate}
	\item Authors? \\
	Elisenda Ortiz Castillo, Ben Curran, Kyle Higham, Demival Vasques Filho
	
	\item Antecipated journals:
	\begin{enumerate}
		
		\item
		
	\end{enumerate}
	
	\item If your readers had only one sentence to summarize your article, what should it be? (25 words) \\ 

	
\end{enumerate}
	

\section{\label{sec:introduction}Introduction}




\section{\label{sec:topic}Topic Modeling}




\section{\label{sec:bipartite}Bipartite networks}



\section{\label{sec:col_features}The parliamentary network}




\section{\label{sec:conclusions}Conclusions}




\begin{acknowledgments}

\end{acknowledgments}

%\bibliographystyle{unsrtnat}
\bibliography{hansardNotes} % The references (bibliography) information are stored in the file named "Bibliography.bib"

\end{document}
%
% ****** End of file apssamp.tex ******


